%
% snowstar.tex -- 
%
% (c) 2016 Prof Dr Andreas Mueller, Hochschule Rapperswil 2016
%
\chapter{Snowstar: the network daemon\label{chapter:snowstar}}
The \texttt{snowstar} daemon is designed to be run on a small
computer mounted directly on the telescope or the mount, and to
control cameras and other devices.
Clients then don't need any knowledge of the intricacies of controlling
hardware devices.

\section{Running the daemon}
The daemon understands the following options:
\begin{itemize}
\item
\texttt{-b,--base=}\textit{imagedir}
\\
The daemon needs a directory where it can temporarily store images until
a client retrieves them and removes them from the daemon's scope.
\item
\texttt{-c,--config}\textit{configdb}
\\
The daemon uses a database for configuration information.
\item
\texttt{-d,--debug}
Increase the debug level.
The debug messages are written to standard output.
\item
\texttt{-D,--database=}\textit{databasefile}
\\
The daemon uses a database to persist data for temporary objects
like tasks in the task queue, instrument configuration, guider
calibrations and guide protocols.
In most cases, the same database is used for configuration and
for persistence.
\item
\texttt{-h,--help}
\\
Display a help message and exist.
\item
\texttt{-f,--foreground}
\\
In normal operation, the daemon goes into the background.
For debugging, it is sometimes more convenient to have it run in the
foreground, which can be forced with this option.
\item
\texttt{-g,--group=}\textit{group}
\\
Change group id to the one of~\textit{group} at run time.
This is most useful when starting the daemon from a start script
that runs as \texttt{root}.
This option must be given before the database options to ensure
that the process runs under the right group id when accessing the
databases.
\item
\texttt{-p,--port=}\textit{port}
\\
Run the service on this port.
There also is a configuration option to configure the TCP port to use
for the service, but this option might be handy during debugging.
\item
\texttt{-P,--pidfile=}\textit{pidfile}
\\
Full path name to the pid file to create when starting up.
The pid file can be used by stop-scripts to get the process id
to of the currently running process.
\item
\texttt{-n,--name=}\textit{servicename}
\\
Run the service with the Zeroconf name \textit{servicename}.
The network clients will use this service name to locate the service.
\item
\texttt{-s,--sslport=}\texttt{sslport}
\\
Run the SSL service on a separate port.
As for the unencrypted port, there is a configuration option to set the
SSL port which is usually preferred.
\item
\texttt{-u,--user=}\textit{user}
\\
Change user id to the one of~\textit{user} at run time.
This is most useful when starting the daemon from a start script
that runs as \texttt{root}.
This option must be given before the database options to ensure
that the process runs under the right group id when accessing the
databases.

\end{itemize}

\section{Configuration}




