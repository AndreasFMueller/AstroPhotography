%
% focusing.tex
%
\chapter{Focusing\label{chapter:focusing}}

\section{Focusing solutions}
All focusing algorithms work by acquiring a number of images at different
focuser positions, compute focus measures for those images and then
find maxima or minima.
Such an algorithm has to be somewhat insensitive to noise in the image,
which can be achieved using some variant of least squares algorithms.

\subsection{Quadratic maximum}
We want to approximate the focus values by a parabola of the form
\[
y=f(x)=a_2x^2+a_1x+a_0,
\]
where $a<0$.
The data points $(x_i,y_i)$ are first selected so that we only work
with a few points close to the maximum, we assume that we have $N$
different points.
For these points we set up the least squares equations
\[
L
=
\sum_{i}(f(x_i)-y_i)^2\to\min.
\]
Differentiating with respect to the parameters $a_k$, we get
\begin{align*}
\frac{\partial L}{\partial a_k}
=
\sum_{i}2(f(x_i) - y_i)\frac{\partial f}{\partial a_k}(x_i)
=
\sum_{i}2(f(x_i)-y_i)x_i^k=0
\end{align*}
This leads to the equations
\begin{align*}
\sum_{i}f(x_i)x_i^k
&=
\sum_{i} y_ix_i^k&& 0\le k\le 2
\\
\sum_{l=0}^2 a_l\sum_{i}x_i^{k+l}
&=
\sum_{i}y_ix_i^k
\end{align*}
Writing the equations for the $a_l$ explicitely, we get the system of
linear equations
\[
\begin{linsys}{3}
(\sum_{i=1}^N 1)a_0&+&
(\sum_{i=1}^N x_i)a_1&+&
(\sum_{i=1}^N x_i^2) a_2 &=& \sum_{i=1}^N y_i
\\
(\sum_{i=1}^Nx_i)a_0&+&
(\sum_{i=1}^N x_i^2)a_1&+&
(\sum_{i=1}^N x_i^3)a_2 &=& \sum_{i=1}^N x_iy_i
\\
(\sum_{i=1}^Nx_i^2)a_0&+&
(\sum_{i=1}^N x_i^3)a_1&+&
(\sum_{i=1}^Nx_i^4)a_2 &=& \sum_{i=1}^N x_i^2y_i
\\
\end{linsys}
\]
The vertex of the parabola is at $x_\text{min}=-a_1/2a_2$, which we might
find using Cramer's rule:
\[
x_{\text{min}}
=
-\frac{
\left|\begin{matrix}
           N       & \sum_{i=1}^N      y_i & \sum_{i=1}^N x_i^2 \\
\sum_{i=1}^N x_i   & \sum_{i=1}^N x_i  y_i & \sum_{i=1}^N x_i^3 \\
\sum_{i=1}^N x_i^2 & \sum_{i=1}^N x_i^2y_i & \sum_{i=1}^N x_i^4 
\end{matrix}\right|
}{
2\left|\begin{matrix}
           N       & \sum_{i=1}^N x_i   & \sum_{i=1}^N      y_i \\
\sum_{i=1}^N x_i   & \sum_{i=1}^N x_i^2 & \sum_{i=1}^N x_i  y_i \\
\sum_{i=1}^N x_i^2 & \sum_{i=1}^N x_i^3 & \sum_{i=1}^N x_i^2y_i 
\end{matrix}\right|
}.
\]

\subsection{Brenner focus measure}
The Brenner focus measures use differences between pixels to measure
the focus quality of an image:
\[
B(f)
=
\sum_{x,y\in \Omega}
\bigl((f(x+1,y)-f(x-1,y))^2
+
(f(x,y+1)-f(x,y-1))^2\bigr),
\]
where $\Omega$ is the image rectangle and $f(x,y)$ is the flux measure
at position $(x,y)$ of the image.
Focus is achieved when $B(f)$ is maximized.

If noise is present in the different images $f_n$ image, then also the
values $B(f_n)$ will be noise, and it can become difficult to find the
maximum.
For this we need a model on how $B(f)$ may depend on the focus position.
To develop such a model, we assume that the out of focus image of a star
is a circle of radius $r$. 
Since the area of the circle is $\pi r^2$, we can model the image by
a function
\[
f_r(x,y)
=
\begin{cases}
\frac1{r^2}&\qquad x^2+y^2\le r^2\\
0          &\qquad\text{otherwise}
\end{cases}
\]
We now estimate the Brenner focus measure $B(f_r)$.
Only points close to the boundary contribute to $B(f_r)$.
There are exactly $4r$ points where $f_r(x-1,y)\ne f_r(x+1,y)$, and
as many points where $f_r(x,y-1)\ne f_r(x,y+1)$.
Also the difference is $1/r^2$, so the Brenner focus measure is
\[
B(f_r)\simeq 8r \frac1{r^4}=\frac8{r^3}.
\]
So we try to fit the values of $B(f_n)$ to a function of the form
\[
b(x)=\frac{a}{|x-c|^3}+d.
\]
Given a set of points $(x_i,y_i)$, we ask for parameter values
$a$, $c$ and $d$ such that
\[
L=\sum_{i=1}^n(b(x_i)-y_i)^2
\]
is minimized.

Computing the derivative 
\begin{align*}
\frac{\partial L}{\partial a}
&=
\sum_{i=1}^n 2(b(x_i)-y_i) \frac{\partial b(x_i)}{\partial a}
=
\sum_{i=1}^n 2(b(x_i)-y_i)\frac{1}{|x_i-c|^3}
\\
\frac{\partial L}{\partial c}
&=
\sum_{i=1}^n 2(b(x_i)-y_i) \frac{\partial b(x_i)}{\partial c}
=
\sum_{i=1}^n 2(b(x_i)-y_i) \frac{3a}{(x_i-c)^4}\operatorname{sign}(x_i-c)
\\
\frac{\partial L}{\partial d}
&=
\sum_{i=1}^n 2(b(x_i)-y_i) \frac{\partial b(x_i)}{\partial d}
=
\sum_{i=1}^n 2(b(x_i)-y_i)
\end{align*}



