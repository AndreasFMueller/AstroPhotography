%
% architectures.tex
%
% (c) 2016 Prof Dr Andreas Müller, Hochschule Rapperswil
%
\chapter{Architecture\label{chapter:architecture}}
\section{Scope}
The AstroPhotography project serves as a test bed to try new algorithms
and architectural ideas in astrophotography software.
It does not attempt to replicate any commonly available functionality
for its own sake, but it provides quite a bit of basic astrophotography
functionalty for new experiments to base on.
E.~g.~almost any code dealing with images will need to be able to
manipulate images, read and write images from files.

The project concentrates on the following areas of interest:
\begin{itemize}
\item Image processing: Algorithms for dark and light frame extraction,
bad pixel interpolation, debayering, image transformation and stacking,
autofocus and background extraction.
\item Guiding: Algorithms for guiding, comprising image analysis, 
calibration of guider port interface, separation of adaptive optics
and mount guiding corrections.
\item Automation: control cameras to automatically perform an series of
exposures.
Automated and maximally parallel image processing.
\item Networking: most modern astrophotograhy software uses a single
computer to control the devices.
This project encourrages use of modern ARM-based single board computers
with WLAN mounted directly on the telescope and mount, complitely eliminating
all cables except power.
To connect them, some network protocol is necessary, and the
AstroPhotography project uses ZeroC Internet Communcation Engine (ICE)
as its networking middleware.
\end{itemize}


