\chapter{Configuration}
The AstroPhotography software has a number of configuration facilities
to be discussed in this chapter.
The main configuration method is the configuration database, most
tools have a command line option ths specify the configuration database
file.
For low level configuration, mostly for device drivers, there is a
system of properties stored in propery files.

\section{Configuration database}
The configuration database contains attribute value pairs that can be
manipulated with the \texttt{astroconfig} utility.
The configruation values are organized into a three-level hierarchy
of \texttt{domain}, \texttt{section} and \texttt{name}, but currently
only the \texttt{global} domain is implemented.

The \texttt{astroconfig} program operates on a configuration database
that can be specified via the \texttt{--config}-command line option.
Other available options of the \texttt{astroconfig} program can be
displayed with the \texttt{--help}-option.

The following subcommands of the \texttt{astroconfig} program all have
a \textit{domain} argument, please not that this currently always has to be 
\texttt{global}.
Here are the commands:
\begin{itemize}
\item
\texttt{get} \textit{domain} \textit{section} \textit{name}
\\
Retrieve the value of the specified configuration variable.
\item
\texttt{set} \textit{domain} \textit{section} \textit{name} \textit{value}
\\
Set the value of the specified configuration variable.
If the variable already exists, it is overwritten.
\item
\texttt{delete} \textit{domain} \textit{section} \textit{name}
\\
Delete the configuration variable from the database.
\item
\texttt{list} \textit{domain} [ \textit{section} [ \textit{name} ] ]
\\
List all variables that match the specified criteria.
If \textit{name} or \textit{section} are not specified, all variables
matching the \textit{domain} or \textit{spection} is displayed.
\end{itemize}

\section{Well-known configuration variables}
This section documents some well known configuration variables from 
various subsystms.
\subsection{Service Location}
The ICE network services use a common service location class that
understands the following global configuration variables
\begin{itemize}
\item
\texttt{service.name}:
The service name to use for the network service.
This name is used in ZeroConf to locate the service.
\item
\texttt{service.port}:
The port to use for the unencrypted service front end.
The default port is 10000.
\item
\texttt{service.sslport}:
The port to use for the TLS-protected service front end.
There is no default for this variable.
\end{itemize}

\subsection{Default repository}
The \texttt{snowimages} program stores images retrieved from a server
in the repository specified by the \texttt{--repository} option.
To simplify usage, it is possible to define default repository
for these operations with the \texttt{repository.default} variable.

When creating a new repository, the top directory within which to create
the repository is specified by \texttt{repository.topdir}.

\subsection{Default project}
The \texttt{snowimages} program also stores a project identifier with
the images it retrieves, but only if the project identifier is specified
with the \texttt{--project} option.
To set the project name for all images, the default project name can
be set as the \texttt{project.default} global configuration variable.


