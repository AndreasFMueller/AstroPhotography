%
% imagerepo.tex -- image repository configuration
%
% (c) 2016 Prof Dr Andreas Mueller, Hochschule Rapperswil
%
\chapter{Image Repositories\label{chapter:imagerepo}}
To simplify access to the metadata of images, this project keeps
images in image repositories. 
This chapter describes how repositories are created, how images
are added, retrieved and removed from a repository.

A repository is essentially a directory containing FITS images and
a database containing the metadata for all the images.
The database allows easier access to the images in the repository.
To ensure consistency between database and image files, repositories
should only be manipulated using the image repository utilities
described below.

\section{Creating and deleting repositories}
The definition of image repositories is considered part of the
configuration of the system, and is therefore implemented in the
\texttt{astroconfig} program.
The program can list repositories, create a repository definition
in the configuration database,
or delete a repository from the configuration database.
It does not manipulate the contents of the image repository directory,
so in particular it does not remove the directory when the repository
is removed.

The \texttt{astroconfig} program accepts the following subcommands:
\begin{itemize}
\item
\texttt{astroconfig imagerepo list}
\\
List all image repositories known to the configuration database.
\item
\texttt{astroconfig imagerepo add} \textit{reponame} \textit{directory}
\\
Create a new repository in directory \textit{directory}.
The information about the repository location is stored in the
configuration database and if necessary, the directory is created.
The database file \texttt{.astro.db} is not created immediately, it will
be created when the repository is first accessed.
\item
\texttt{astroconfig imagerepo remove} \textit{reponame}
\\
This command removes the repository definition from configuration
database.
It does not remove the images, the metadata database or the directory.
\end{itemize}

\section{Working within repositories}
The \texttt{imagerepo} program is used to manipulate the contents of
an image repository.

\begin{itemize}
\item
List the contents of the repository:
\\
\texttt{imagerepo} \textit{reponame} \texttt{list}
\item
Add an image to the repository:
\\
\texttt{imagerepo} \textit{reponame} \texttt{add} \textit{image.fits}
\\
This command copies the image file \textit{image.fits} to the
image repository directory, extracts the metadata from the image and
adds it to the repository.
If the image did not contain a UUID yet, a UUID will be added to the
copy.
\item
Show information about an image in the repository:
\\
\texttt{imagerepo} \textit{reponame} \texttt{show}
\item
Retrieve an image from the repository:
\\
\texttt{imagerepo} \textit{reponame} \texttt{get} \textit{imageid} \textit{image.fits}
\\
This command extracts the image with id \textit{imageid} and write it to
the file \textit{image.fits}.
\item
Remove an image from the repository:
\\
\texttt{imagerepo} \textit{reponame} \texttt{remove} \textit{imageid}
\end{itemize}

\section{Working with two repositories}
The \texttt{imagerepo} program also allows to move images between
two repositories.
The main idea is that it should be possible to move images in a consistent
way between images on a removable medium and a local hard drive.
To accomplish this, there are low level commands to copy or move images
from one repository to the other.
These commands are
\begin{itemize}
\item
\texttt{imagerepo} \textit{srcrepo} \texttt{copy} \textit{imageid} \textit{targetrepo}
\\
Copy image with id \textit{imageid} from the image repository named
\texttt{srcrepo} to the image repository named \textit{targetrepo}.
The image will have a new image id in the target repository, but all
other metadata is conserved, in particular the UUID of the image is
unchanged.
\item
\texttt{imagerepo} \textit{srcrepo} \texttt{move} \textit{imageid} \textit{targetrepo}
\\
Like the copy command, but after a successful move, the image is deleted
in the source image repository.
\end{itemize}
In most cases, an imaging session creates a large number of new images
in a repository, and it can be quite tedious to copy the right images
to the target repository. 
This is where the replication and synchronization commands come in:
\begin{itemize}
\item
\texttt{imagerepo} \textit{srcrepo} \texttt{replicate} \textit{targetrepo}
\\
Replicate all images from \textit{srcrepo} that are not already present
in the \textit{targetrepo} into the target repository.
\item
\texttt{imagerepo} \textit{srcrepo} \texttt{synchronize} \textit{targetrepo}
\\
Synchronize all images between two repositories.
This is essentially a two-way replication command.
\end{itemize}
The synchronization command can be used to keep a central, e.~g.~NAS-based
image repository of images in sync with a local repository.

